Since PaMPa-HD is able to process extremely high-dimensional datasets we believe it is suitable for 
many application (scientific) domains. 
The first example is bioinformatics: 
researchers in this environment often cope with data structures 
defined by a large number of attributes, 
which matches gene expressions, 
and a relatively small number of transactions, 
which typically represent medical patients or tissue samples. 
Furthermore, smart cities and computer vision environments 
are two important application domains which can benefit 
from our distributed algorithm,
thanks to their heterogeneous nature.
Another field of application is the networking domain. 
%This environment is surely the one with the major amount and types of collected data. 
%The reason is related to the high number of available information sources and the easiness to collect information. 
Some examples of interesting high-dimensional dataset are 
URL reputation, advertisements, social networks and search engines. 
One of the most interesting applications, 
which we plan to investigate in the future,
is related to internet traffic measurements. 
Currently, the market offers an interesting variety of internet packet sniffers like \cite{Tstat}, \cite{netflow}. 
Datasets, in which the transactions represent flows and the item are flows attributes, 
are already a very promising application domain 
for data mining techniques \cite{trustcom2013},\cite{fontas_AR}, \cite{Netmine}. 
%These datasets are characterized by a large number of transactions (usually millions per hour) and few tens of attributes. 
%However, we plan to merge all the transactions within a time window and apply some data mining techniques, 
%such as our Distributed Carpenter implementation. 
%The target would be to extract some deeply hidden knowledge, if there is, related to network status, 
%trying to early detect or predict anomalous events or congestions. 
