% %Frequent closed itemset mining is among the most complex exploratory techniques
% in data mining,
% %and provides the ability to discover hidden correlations in transactional
% datasets.
% %The explosion of Big Data is leading to new parallel and distributed
% approaches.
% %Unfortunately, most of them are designed to cope with low-dimensional datasets,
% %whereas no distributed high-dimensional frequent closed itemset mining
% algorithms exists.
% %This work introduces PaMPa-HD,
% %a parallel MapReduce-based frequent closed itemset mining algorithm
% %for high-dimensional datasets, based on Carpenter.
% %The experimental results, performed on both real and synthetic datasets,
% %show the efficiency and scalability of PaMPa-HD.
% %PaM-Car outperforms both the single-machine Carpenter implementation
% %and the best state-of-the-art distributed approaches

%Frequent closed itemset mining, a data mining technique for discovering hidden
%correlations in transactional datasets, is among the most complex exploratory
%techniques in data mining.
%Thanks to the spread of distributed and
%parallel frameworks, the development of scalable approaches able to deal with
%the so called Big Data has been extended to frequent itemset mining.
%Unfortunately, most of the current algorithms are designed to cope with
%low-dimensional datasets,
%delivering poor performances in those use cases characterized by
%high-dimensional data.
%This work introduces PaMPa-HD, a parallel MapReduce-based frequent closed
%itemset mining algorithm for high dimensional datasets.
%The experimental results, performed on two real-life high-dimensional use cases,
%show the efficiency of the proposed approach in terms of execution time, load balancing and robustness to memory issues.




In today's world, large volumes of data are being continuously
generated by many scientific applications, such as bioinformatics or networking.
Since each monitored event is usually characterized by a variety of features,
high-dimensional datasets have been continuously generated. 
To extract value from these complex collections of data, different exploratory data mining algorithms
can be used to discover hidden and non-trivial correlations among data.
Frequent closed itemset mining is an effective but computational expensive 
technique that is usually used to support data exploration. 
Thanks to the spread of distributed and
parallel frameworks, the development of scalable approaches able to deal with
the so called Big Data has been extended to frequent itemset mining.
Unfortunately, most of the current algorithms are designed to cope with
low-dimensional datasets,
delivering poor performances in those use cases characterized by
high-dimensional data.
This work introduces PaMPa-HD, a MapReduce-based frequent closed
itemset mining algorithm for high dimensional datasets.
An efficient solution has been proposed to 
parallelize and speed up the mining process. Furthermore, different strategies have been proposed to easily configure the algorithm parameter.
The experimental results, performed on real-life high-dimensional use cases,
show the efficiency of the proposed approach in terms of execution time, load balancing and robustness to memory issues.
