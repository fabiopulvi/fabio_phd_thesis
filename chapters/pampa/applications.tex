%\textbf{Questo tutto vecchio, rifrasare? Lo eliminiamo? Michiardi: solo se
%abbiamo problemi di spazio} Since PaMPa-HD is able to process extremely
%high-dimensional datasets we believe it is suitable for
%many application (scientific) domains.
Since PaMPa-HD is able to process extremely
high-dimensional datasets, it enriches the set of algorithm
able to deal with datasets characterized by a very large variety of features (e.g.~\cite{Vimieiro20141},~\cite{Bermejo201235}).
Consequentely, many fields of applications which exploits frequent itemset to discover hidden correlations and association rules~\cite{KamsuFoguem20131034}
 could benefit of it.
The first example is bioinformatics~\cite{Nahar20131086} and health environments:
researchers in this domain often cope with data structures
defined by a large number of attributes,
which matches gene expressions,
and a relatively small number of transactions,
which typically represent medical patients or tissue samples.
Furthermore, smart cities and computer vision applications
are two important domains which can benefit
from our distributed algorithm,
thanks to their heterogeneous nature.
Another field of application is the networking domain.
%This environment is surely the one with the major amount and types of
%collected data.
%The reason is related to the high number of available information sources
%and the easiness to collect information.
Some examples of interesting high-dimensional dataset are
URL reputation, advertisements, social networks and search engines.
One of the most interesting applications,
which we plan to investigate in the future,
is related to internet traffic measurements.
Currently, the market offers an interesting variety of internet packet sniffers
like~\cite{Tstat},~\cite{netflow}. Collected datasets, that include traffic flows in which the item are flow
attributes (\cite{trustcom2013},~\cite{fontas_AR},~\cite{Netmine}), represent an appealing domain where
PaMPa-HD can be efficiently exploited.
are already a very promising application domain
for data mining techniques.
% %These datasets are characterized by a large number of transactions (usually
% millions per hour) and few tens of attributes.
% %However, we plan to merge all the transactions within a time window and apply
% some data mining techniques,
% %such as our Distributed Carpenter implementation.
% %The target would be to extract some deeply hidden knowledge, if there is,
% related to network status,
% %trying to early detect or predict anomalous events or congestions.
