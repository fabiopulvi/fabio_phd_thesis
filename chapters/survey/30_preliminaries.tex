
\begin{figure}[!t]
%\renewcommand{\arraystretch}{1.3}
%\centerline

{\subfloat[Horizontal representation of $\mathcal{D}$]{
\label{horizontalexampledataset}
\begin{tabular}{|c|l|}

\hline
\multicolumn{2}{|c|}{$\mathcal{D}$}\\
\hline
\hline
	tid & items \\
\hline
	1 & a,b,c,l,o,s,v \\
\hline
	2 & a,d,e,h,l,p,r,v \\
\hline
	3 & a,c,e,h,o,q,t,v \\
\hline
	4 & a,e,f,h,p,r,v \\
\hline
	5 & a,b,d,f,g,l,q,s,t \\
\hline
\end{tabular}}}%
\hfil
{\subfloat[Vertical representation of $\mathcal{D}$]{
\label{TTexampledataset}
\begin{tabular}{|l|l|}
\hline
\multicolumn{2}{|c|}{Vertical Representation}\\
\hline
\hline
	item & tidlist \\ \hline
	a & 1,2,3,4,5 \\ \hline
	b & 1,5 \\ \hline
	c & 1,3 \\ \hline
	d & 2,5 \\ \hline
	e & 2,3,4 \\ \hline
	f & 4,5 \\ \hline
	g & 5 \\ \hline
	h & 2,3,4 \\ \hline
	l & 1,2,5 \\ \hline
	o & 1,3 \\ \hline
	p & 2,4 \\ \hline
	q & 3,5 \\ \hline
	r & 2,4 \\ \hline
	s & 1,5 \\ \hline
	t & 3.5 \\ \hline
	v & 1,2,3,4 \\ \hline
\end{tabular}}}%

\caption{Running example dataset $\mathcal{D}$}
\label{exampledataset}
\end{figure}
In this section, a basic introduction to frequent itemset mining will
be given to the readers.
Let $\mathcal{I}$ be a set of items. A transactional dataset $\mathcal{D}$
consists of a set of transactions $\{t_1, \dots, t_n\}$.
Each transaction $t_i\in \mathcal{D}$ is a collection of items
(i.e., $t_i\subseteq \mathcal{I}$)
and it is identified by a transaction identifier ($tid_i$).
Figure~\ref{horizontalexampledataset} reports an example of a transactional
dataset with 5 transactions.

% %The dataset reported in Figure~\ref{horizontalexampledataset} is used as a
% running example through the paper.
% %
% %An itemset $I$ is defined as a set of items (i.e., $I\subseteq\mathcal{I}$) and
% it is characterized by a tidlist and a support value.
% %The tidlist of an itemset $I$, denoted by $tidlist(I)$, is defined as the set
% of tids of the transactions in $\mathcal{D}$ containing $I$,
% %while the support of $I$ in $\mathcal{D}$, denoted by $sup(I)$, is defined as
% the ratio between the number of transactions in $\mathcal{D}$ containing $I$
% %and the total number of transactions in $\mathcal{D}$ (i.e.,
% $|tidlist(I)|/|\mathcal{D}|$).
% %For instance, the support of the itemset \textit{\{aco\}} in
% %the running example dataset $\mathcal{D}$ is 2/5 and its tidlist is $\{1,3\}$.
% %An itemset $I$ is considered frequent if its support is greater than a
% user-provided minimum support threshold $minsup$.

An itemset $I$ is defined as a set of items (i.e., $I\subseteq\mathcal{I}$)
and it is characterized by a support value, which is denoted by $sup(I)$ and
defined as the ratio between the number of transactions in $\mathcal{D}$
containing $I$ and the total number of transactions in $\mathcal{D}$.
%(i.e., $|tidlist(I)|/|\mathcal{D}|$).
In the example dataset in Figure~\ref{horizontalexampledataset}, for instance,
the support of the itemset \textit{\{aco\}} is 2/5. . This value represents the frequency of occurrence of the itemset in the dataset.

Given a transactional dataset $\mathcal{D}$ and a minimum support
threshold $minsup$, the Frequent Itemset Mining \cite{KumarBook} problem
consists in extracting the complete set of frequent itemsets
from $\mathcal{D}$.



Many subsets of frequent itemsets exist.
In this paper, we focus on closed itemsets.
Closed itemsets~\cite{ClosedPasquier1999} are a
particular and valuable subset of frequent itemsets, being
a concise but complete representation of frequent itemsets. 
Precisely, an itemset $I$ is closed if none of its supersets (i.e. the set of itemsets which include $I$) has the same support count as $I$.

%If an itemset is frequent, all of its subsets are frequent for the \textit{monotonic property}.
%For the same rule, if an items or an itemset is not frequent, none of its supersets are frequent. In addition, an items
% or itemset $I$ is closed in  $\mathcal{D}$ if there exists no superset that has the same support count as  $I$.\\

A transactional dataset can also be represented in a vertical format, which is
usually a more effective representation for datasets characterised by an average
number of items per transaction orders of magnitudes larger than the number of
transactions.
In this representation, each
row consists of an item $i$ and the list of transactions in which it appears,
also called $tidlist(\{i\})$.
For instance, the tidlist of the itemset \textit{\{aco\}} in
the example dataset $\mathcal{D}$ is $\{1,3\}$.
Figure~\ref{TTexampledataset} reports the transposed representation of the
running example reported in Figure~\ref{horizontalexampledataset}. The main
advantage of the vertical format is the possibility to obtain the tidlist of
an itemset just intersecting the tidlists of the included items, without the
need of a full scan of the dataset.


