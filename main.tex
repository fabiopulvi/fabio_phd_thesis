%% 
%% Copyright 2007, 2008, 2009 Elsevier Ltd
%% 
%% This file is part of the 'Elsarticle Bundle'.
%% ---------------------------------------------
%% 
%% It may be distributed under the conditions of the LaTeX Project Public
%% License, either version 1.2 of this license or (at your option) any
%% later version.  The latest version of this license is in
%%    http://www.latex-project.org/lppl.txt
%% and version 1.2 or later is part of all distributions of LaTeX
%% version 1999/12/01 or later.
%% 
%% The list of all files belonging to the 'Elsarticle Bundle' is
%% given in the file `manifest.txt'.
%% 

%% Template article for Elsevier's document class `elsarticle'
%% with numbered style bibliographic references
%% SP 2008/03/01

\documentclass[preprint,review,12pt]{elsarticle}

%% Use the option review to obtain double line spacing
%% \documentclass[authoryear,preprint,review,12pt]{elsarticle}

%% Use the options 1p,twocolumn; 3p; 3p,twocolumn; 5p; or 5p,twocolumn
%% for a journal layout:
%% \documentclass[final,1p,times]{elsarticle}
%% \documentclass[final,1p,times,twocolumn]{elsarticle}
%% \documentclass[final,3p,times]{elsarticle}
%% \documentclass[final,3p,times,twocolumn]{elsarticle}
%% \documentclass[final,5p,times]{elsarticle}
%% \documentclass[final,5p,times,twocolumn]{elsarticle}

%% For including figures, graphicx.sty has been loaded in
%% elsarticle.cls. If you prefer to use the old commands
%% please give \usepackage{epsfig}

%% The amssymb package provides various useful mathematical symbols
\usepackage{amssymb}
\usepackage[caption=false,font=footnotesize]{subfig}
\usepackage{multirow}
\usepackage{capt-of}
\usepackage{algpseudocode}
\usepackage{algorithm}
%\usepackage{caption}


  
 % \usepackage{cite}
%% The amsthm package provides extended theorem environments
%% \usepackage{amsthm}

%% The lineno packages adds line numbers. Start line numbering with
%% \begin{linenumbers}, end it with \end{linenumbers}. Or switch it on
%% for the whole article with \linenumbers.
%% \usepackage{lineno}



\begin{document}

\begin{frontmatter}

%% Title, authors and addresses

%% use the tnoteref command within \title for footnotes;
%% use the tnotetext command for theassociated footnote;
%% use the fnref command within \author or \address for footnotes;
%% use the fntext command for theassociated footnote;
%% use the corref command within \author for corresponding author footnotes;
%% use the cortext command for theassociated footnote;
%% use the ead command for the email address,
%% and the form \ead[url] for the home page:
%% \title{Title\tnoteref{label1}}
%% \tnotetext[label1]{}
%% \author{Name\corref{cor1}\fnref{label2}}
%% \ead{email address}
%% \ead[url]{home page}
%% \fntext[label2]{}
%% \cortext[cor1]{}
%% \address{Address\fnref{label3}}
%% \fntext[label3]{}

\title{Scalable Frequent Itemset Mining algorithms for Big Data}

%% use optional labels to link authors explicitly to addresses:
%% \author[label1,label2]{}
%% \address[label1]{}
%% \address[label2]{}

\author{Fabio Pulvirenti}


\address{Dipartimento di Automatica e Informatica\\
Politecnico di Torino\\
Torino, Italy\\ Email: name.surname@polito.it}

\begin{abstract}
%% Text of abstract

\end{abstract}

\begin{keyword}

%% keywords here, in the form: keyword \sep keyword

%% PACS codes here, in the form: \PACS code \sep code

%% MSC codes here, in the form: \MSC code \sep code
%% or \MSC[2008] code \sep code (2000 is the default)

\end{keyword}

\end{frontmatter}

%% \linenumbers

%% main text
\section{Introduction}
In the last years, the increasing capabilities of recent applications
to produce and store huge amounts of information,
the so called "Big Data"~\cite{Jin201559}, have changed dramatically
the importance of the intelligent analysis of data.
Data mining, together with machine learning~\cite{DBLP:journals/bdr/Al-JarrahYMKT15}, 
is considered
one of the fondamental tools on which Big Data analytics
are based.
In both academic and industrial domains, the interest towards data mining,
which focuses on extracting effective and usable knowledge from large
collections of data, has risen.
The need for efficient and highly scalable data mining tools increases with the
size of the datasets,
as well as their value for businesses and researchers aiming at extracting
meaningful insights increases.\\
Frequent (closed) itemset mining is among the most complex exploratory
techniques in data mining.
It is used to discover frequently co-occurring items
according to a user-provided frequency threshold, called minimum support.
Existing mining algorithms revealed to be very efficient on simple datasets
but very resource intensive in Big Data contexts.
In general, the application of data mining techniques to Big Data collections
is characterized by the need of huge amount of resources.
For this reason, we are witnessing the explosion of parallel and distributed
approaches,
typically based on distributed frameworks, such as Apache Hadoop~\cite{HDFS}
and Spark~\cite{Zaharia_spark}.
As clearly shown in Chapter \ref{survey}, unfortunately, most of the scalable distributed techniques
for frequent itemset mining have been designed to cope with datasets
characterized by few items per transaction (low dimensionality, short
transactions). Their design, on the contrary, focuses on very large datasets in terms of number of
transactions.
Currently, only single-machine implementations exist to address very long
transactions,
such as Carpenter~\cite{Zaki_Carpenter}, and no distributed implementations at
all.\\
%Nevertheless, many researchers in scientific domains such as bioinformatics or
%networking,
%often require to deal with this type of data.
Nevertheless, many scientific applications, such as bioinformatics or networking, 
generate a large number of events characterized by a variety of features.
Thus, high-dimensional datasets have been continuosly generated.
For instance, most gene expression datasets are characterized by
a huge number of items (related to tens of thousands of genes)
and a few records (one transaction per patient or tissue).
Many applications in computer vision deal with high-dimensional data, such as
face recognition.
An increasing portion of big data is actually related to geospatial data~\cite{Lee201574} and
smart-cities. Some studies have built this type of large datasets
measuring the occupancy of different car lanes:
each transaction describes the occupancy rate in a captor location and in a
given timestamp~\cite{PEMSDataset}.
In the networking domain, instead,
the heterogeneous environment provides many different datasets
characterized by high-dimensional data,
such as URL reputation, advertising, and social network
datasets~\cite{snapnets}.
To effectively deal with those high-dimensional datasets,
novel and distributed approaches are needed.

This work introduces PaMPa-HD \cite{pampa_v1}, \cite{pampa_pulvi},
a parallel MapReduce-based frequent closed itemset mining algorithm
for high-dimensional datasets.
PaMPa-HD relies on the Carpenter algorithm~\cite{Zaki_Carpenter}. 
The PaMPa-HD design\footnote{The source code of PaMPa-HD can be downloaded from https://github.com/fabiopulvi/PaMPa-HD},through an ad-hoc synchronization technique, takes into account crucial design aspects,
such as load balancing and robustness to memory-issues. Furthermore, different strategies have been proposed to easily tune up the parameter configuration.
The algorithm has been thoroughly evaluated on real high dimensional datasets. 
PaMPa-HD outperforms the state-of-the-art distributed approaches
in execution time and by supporting lower minimum support threshold. 


The paper is organized as follows:
Section~\ref{Preliminaries} briefly reintroducess the frequent (closed) itemset mining
problem,
Section~\ref{Carpenter algorithm} briefly describes the centralized version
of Carpenter,
and Section~\ref{Distributed implementation outline} presents the proposed
PaMPa-HD algorithm.
Section~\ref{Experiments} describes the experimental evaluations
proving the effectiveness of the proposed technique,
Section~\ref{Applications} discusses possible applications of PaMPa-HD and, finally, Section~\ref{Conclusion} introduces future works and conclusions.
%Section~\ref{Related work} presents a brief review of the state of the art,
%and Section~\ref{Applications} discusses possible applications of PaMPa-HD.
%Finally, Section~\ref{Conclusion} introduces future works and conclusions.



\label{Introduction}



\section{Problem statement}
\begin{enumerate}
\item Why FIM for Big Data
\item Which are the challenges
\end{enumerate}
\section{Related works - Survey}
Analysis of the state of the art
\section {Frequent Itemset Mining for high dimensional data}
PaMPa-HD
\section{Applications of Frequent Itemset Mining to distributed frameworks}
\begin{enumerate}
\item{MGI-Cloud}
\item{Nemico}
\end{enumerate}




\section{Conclusion} \label{Conclusion}





\bibliographystyle{elsarticle-num_modified}
\bibliography{biblio}{}

%% \bibitem{label}
%% Text of bibliographic item



\end{document}
%\endinput
%%
%% End of file `elsarticle-template-num.tex'.
\grid
